\documentclass[aspectratio=169,11pt]{beamer}

% ============================================
% SPEAKER NOTES
% ============================================
% Compile options (uncomment ONE):
%   - For projection slides: leave both commented (default hides notes)
%   - For podium notes PDF: uncomment 'show only notes'
%
% \setbeameroption{hide notes}       % Default: slides only
%\setbeameroption{show only notes}  % Notes PDF for podium


\usetheme{default}
\usecolortheme{default}

% Define custom colors for highlighting
\definecolor{highlight}{RGB}{0,102,204}      % Professional blue for highlighting
\definecolor{emphasis}{RGB}{204,0,0}         % Red for strong emphasis

% Create easy-to-use commands
\newcommand{\highlight}[1]{\textcolor{highlight}{#1}}
\newcommand{\redemph}[1]{\textcolor{emphasis}{\textbf{#1}}}

\definecolor{headerblue}{RGB}{51,51,153}
\definecolor{darkblue}{RGB}{51,51,153}

% Frame title: blue rounded box with white text
\setbeamertemplate{frametitle}{%
  \vspace{0.3cm}
  \begin{beamercolorbox}[wd=\paperwidth,ht=0.8cm,dp=0.3cm,leftskip=0.3cm]{frametitle}
    \usebeamerfont{frametitle}\insertframetitle
  \end{beamercolorbox}
}
\setbeamercolor{frametitle}{fg=white,bg=headerblue}
\setbeamerfont{frametitle}{size=\large}

% Title slide colors
\setbeamercolor{title}{fg=white}
\setbeamercolor{subtitle}{fg=white}
\setbeamercolor{author}{fg=black}
\setbeamercolor{institute}{fg=black}
\setbeamercolor{date}{fg=black}

% Title page: blue box for title
\setbeamertemplate{title page}{
  \vspace{1cm}
  \begin{beamercolorbox}[wd=\paperwidth,ht=1.5cm,dp=0.5cm,center]{title}
    \usebeamerfont{title}\inserttitle\\[0.2cm]
    \usebeamerfont{subtitle}\insertsubtitle
  \end{beamercolorbox}
  \vspace{0.8cm}
  \begin{center}
    \usebeamercolor[fg]{author}\usebeamerfont{author}\insertauthor\\[0.3cm]
    \usebeamercolor[fg]{institute}\usebeamerfont{institute}\insertinstitute\\[0.3cm]
    \usebeamercolor[fg]{date}\usebeamerfont{date}\insertdate
  \end{center}
}
\setbeamercolor{title}{bg=headerblue}

% Structure color for itemize bullets
\setbeamercolor{structure}{fg=darkblue}

% Button colors for navigation
\setbeamercolor{button}{bg=darkblue,fg=white}
\setbeamercolor{button border}{fg=darkblue}

% Small filled circle bullets (like Shapiro)
\setbeamertemplate{itemize item}{\small$\bullet$}
\setbeamertemplate{itemize subitem}{\small$\bullet$}
\setbeamertemplate{itemize subsubitem}{\small$\bullet$}
\setbeamertemplate{enumerate items}[default]

% Minimal footer: subsection (linked to outline) left, frame numbers right
\setbeamertemplate{navigation symbols}{}
\setbeamertemplate{footline}{%
  \hbox to \paperwidth{%
    \hspace{0.3cm}%
    \hyperlink{toc}{\textcolor{gray!70}{\scriptsize$\triangleleft$~\insertsubsection}}%
    \hfill%
    \textcolor{gray}{\scriptsize\insertframenumber/\inserttotalframenumber}%
    \hspace{0.3cm}%
  }%
  \vspace{0.2cm}%
}

% No headline
\setbeamertemplate{headline}{}

% Roadmap slide command: shows TOC with current section highlighted
\newcommand{\roadmapslide}{%
  \begin{frame}[plain]
    \vfill
    \begin{center}
      \tableofcontents[currentsection,hideallsubsections]
    \end{center}
    \vfill
  \end{frame}
}

% ============================================
% PACKAGES
% ============================================
\usepackage{amsmath,amssymb}
\usepackage{graphicx}
\usepackage{booktabs}
\usepackage{hyperref}
\hypersetup{
  colorlinks=false,
  linkcolor=darkblue,
  urlcolor=darkblue
}
\usepackage{tikz}
\usetikzlibrary{shapes,arrows,positioning}
\usepackage{pgfpages}  % Required for notes functionality

% TOC styling - use beamer's built-in linking
\setbeamercolor{section in toc}{fg=darkblue}
\setbeamercolor{subsection in toc}{fg=gray!80}

% ============================================
% TITLE INFORMATION
% ============================================
\title[Vertical Disintegration \& Farm Productivity]{Vertical Disintegration and the Shifting Boundary of the Farm Business}
\subtitle{Implications for Agricultural Productivity}

\author[Brent Hueth]{Brent Hueth}

\institute[]{
  USDA Economic Research Service\\
  \smallskip
  \textit{Joint work with Anton Babkin (UW-Madison) and Richard A. Dunn (Sandhill Consulting)}
}

\date[Job Seminar]{\today}

% ============================================
% DOCUMENT
% ============================================
\begin{document}

% --------------------------------------------
% TITLE SLIDE
% --------------------------------------------
\begin{frame}[plain]
\titlepage
\end{frame}

% ============================================
% OUTLINE
% ============================================

\begin{frame}{Outline}
\label{toc}
\hypertarget{toc}{}
\tableofcontents
\end{frame}

% ============================================
% SECTION 1: RESEARCH OVERVIEW
% ============================================

\section{Research Overview: Past, Present, and Future}
\roadmapslide

% --------------------------------------------
\subsection{Past and Present}
\label{subsec:past}

\begin{frame}{Research Background}

\textbf{Agricultural contracting and vertical coordination (1999-2007)}
\begin{itemize}
    \item Contract design, risk allocation, quality measurement in agricultural markets
    \item \textit{AJAE} (1999); \textit{RAE} (1999); \textit{ERAE} (2002)
\end{itemize}

\medskip

\textbf{Cooperative organization and governance (2008-2020)}
\begin{itemize}
    \item Economics of member-owned firms: incentives, monitoring, entry
    \item \textit{AJAE} (2009); \textit{Econ. Letters} (2014); \textit{J. Econ. \& Mgmt. Strategy} (2015)
\end{itemize}

\medskip

\textbf{Cooperative business, supply chains, productivity measurement (2010-present)}
\begin{itemize}
   \item Cooperative business census
   \item Founded UW-Madison FSRDC
   \item CNSTAT Complex Farms report
   \item \textit{AJAE} (2017); \textit{AEPP} (2022); \textit{AEPP} (2023)
\end{itemize}

\medskip

\textit{Today's paper connects all three themes: examining how organizational change affects agricultural productivity using novel administrative data.}

\note{
  \begin{itemize}
    \item 5 minutes on the past, 10 minutes on the future --- Sampling of past work
    \item Producer price risk and quality measurement, stylized fact, foundation for risk management --- Incentive instruments, description from survey data --- Efficient contract, structural estimation
    \item CEOs and incentive pay --- Missing markets and entry, parts 1 and 2
    \item Brief story and background on this path; exposed to and motivated by major data gaps relevant to the field and center
  \end{itemize}
}

\end{frame}

% --------------------------------------------
\subsection{Future}
\label{subsec:future}

% THEME 1: Understanding the Sector
\begin{frame}{Future Research: Understanding the Sector}

\textbf{Cooperatives are central to my research agenda for this position.}

\bigskip

\textbf{Building a comprehensive picture of cooperative business:}
\begin{itemize}
    \item Census of cooperatives---systematic data collection and description
    \item Historical context: how did we get here? (DHIA, Farm Credit, dairy industry development)
    \item Trends and patterns as foundation for education and outreach
\end{itemize}

\bigskip

\textbf{Why this matters:}
\begin{itemize}
    \item Cooperatives are understudied relative to their economic importance
    \item Good data enables good research and informed policy
    \item Foundation for the other two themes
\end{itemize}

\note{
  \begin{itemize}
    \item Signal commitment to cooperatives as central focus
    \item Today's paper is supply chain focused---reflects ERS work
    \item But cooperatives are where I want to build my program here
    \item Census work: building the descriptive foundation
  \end{itemize}
}

\end{frame}

% THEME 2: Governance and Management
\begin{frame}{Future Research: Governance and Management}

\textbf{Inside the cooperative firm:}

\bigskip

\textbf{Board behavior and decision-making}
\begin{itemize}
    \item How do cooperative boards function? What drives their decisions?
    \item Managing member heterogeneity---balancing diverse interests
    \item Implications for leadership training and development
\end{itemize}

\bigskip

\textbf{Capital structure challenges}
\begin{itemize}
    \item The ``curious case'' of cooperative capital
    \item Equity redemption, retained earnings, and growth constraints
    \item Financial sustainability in a changing agricultural landscape
\end{itemize}

\note{
  \begin{itemize}
    \item Tell board behavior story---what we've learned, what we don't know
    \item Curious case of capital---unique challenges of cooperative finance
    \item Core subject matter for extension and leadership training
    \item Direct connection to ISU's cooperative education mission
  \end{itemize}
}

\end{frame}

% THEME 3: Markets and Policy
\begin{frame}{Future Research: Markets and Policy}

\textbf{Cooperatives in market context:}

\bigskip

\textbf{Industrial organization perspectives}
\begin{itemize}
    \item Role of cooperatives in market structure and competition
    \item Entry, exit, and equilibrium in markets with cooperative firms
    \item Implications for producer welfare and market power
\end{itemize}

\bigskip

\textbf{Policy and development}
\begin{itemize}
    \item Formation and growth of new cooperative organizations
    \item Public policy support for cooperative development
    \item Lessons from historical and international experience
\end{itemize}

\note{
  \begin{itemize}
    \item Tell Missouri story---real-world policy relevance
    \item IO perspective: how do cooperatives affect market outcomes?
    \item Development angle: how do new cooperatives form and succeed?
    \item Policy implications for USDA, state programs
  \end{itemize}
}

\end{frame}



% --------------------------------------------
\begin{frame}{Future Research: Complementary Themes}

\textbf{1. Agricultural supply chains and upstream industries}
\begin{itemize}
    \item Farm consolidation effects on input supply and service markets
    \item Role of cooperatives in this context; farm driven distintegration
\end{itemize}

\bigskip

\textbf{2. Productivity measurement}
\begin{itemize}
    \item Quality adjustment for service inputs in TFP accounting
    \item Market power and implications for ag productivity
\end{itemize}

\note{
  \begin{itemize}
    \item Complementary in sense of connecting with broader themes and context
    \item This section focuses on research with supply chain connection
    \item Disintegration term happens alongside consolidation; emphasis on consequences upstream
    \item Based on work initiated while at ERS
    \item Great jumping off point, pivot to paper presentation
  \end{itemize}
}

\bigskip

The paper I'll present next addresses both themes---examining how upstream services affect farm productivity while highlighting measurement challenges.

\end{frame}

% ============================================
% SECTION 2: PAPER PRESENTATION
% ============================================

\section{Farm Vertical Disintegration and Productivity}
\roadmapslide

% --------------------------------------------
\begin{frame}{Disclaimers}

\small

\textbf{USDA Disclaimer:}\\
This research is supported by the U.S. Department of Agriculture, Economic Research Service. The findings and conclusions in this publication are those of the author(s) and should not be construed to represent any official USDA or U.S. Government determination or policy.

\bigskip

\textbf{Census Bureau Disclaimer:}\\
Any views expressed are those of the authors and not those of the U.S. Census Bureau. The Census Bureau has reviewed this data product to ensure appropriate access, use, and disclosure avoidance protection of the confidential source data used to produce this product.

\end{frame}

% --------------------------------------------
\subsection{Motivation}
\label{subsec:motivation}

%SLIDE 1: The Hook - TFP puzzle
\begin{frame}{A Puzzle in American Agriculture}

\begin{columns}[c]
\begin{column}{0.55\textwidth}
\centering
\includegraphics[width=\textwidth]{figures/productivity2021.pdf}
\end{column}

\begin{column}{0.42\textwidth}
\textbf{The productivity puzzle:}

\bigskip

\begin{itemize}
    \item Output has \highlight{nearly tripled} since 1948

    \bigskip

    \item Aggregate inputs are \highlight{essentially flat}

    \bigskip

    \item TFP growth: \highlight{1.49\% annually}
\end{itemize}

\bigskip
\bigskip

\redemph{\textit{Where does this productivity come from?}}

\end{column}
\end{columns}

\note{
  \begin{itemize}
    \item Hook: Start with a puzzle, not a literature review
    \item ERS productivity accounts---output nearly 3x, inputs flat
    \item Are inputs really flat?!!!
    \item Standard story: R\&D, technology, input quality
    \item Huge literature that decomposes TFP into scale, efficiency, and trend effects
    \item Why do we even construct this measure if sits on top of output
    \item Personally, I think we should be focusing on firm-level consolidation, and
    \item Upstream industries
    \item But what about organizational change?
  \end{itemize}
}

\end{frame}

%SLIDE 2: The disintegration trend
\begin{frame}{A Clue: The Changing Structure of Farm Work}

\begin{columns}[c]
\begin{column}{0.55\textwidth}
\centering
\includegraphics[width=\textwidth]{figures/pct_change_2002.pdf}

\smallskip
{\scriptsize\textcolor{gray}{Source: Economic Census \& Census of Agriculture}}
\end{column}

\begin{column}{0.42\textwidth}
\textbf{Two striking trends since (at least) 2002:}

\medskip

\begin{itemize}
    \item Farms are consolidating: \highlight{$\downarrow$ 40\%}

    \smallskip

    \item Upstream industries employment is growing: \highlight{$\uparrow$ 25\%}
\end{itemize}

\medskip

{\small\textbf{Upstream industries:} support actvities, fert and pest manuf., equipment manuf., input wholesalers}

\bigskip

{\small\textit{Work is moving off the farm---\redemph{disintegration} \\ Could this contribute to productivity growth?}}
\end{column}
\end{columns}

\note{
  \begin{itemize}
    \item Data: Economic Census + Census of Agriculture, 2002--2017
    \item Upstream NAICS: 115112--115116 (custom farm services), 115210 (farm mgmt), 325320 (fertilizer), 333111 (tractors), 424910 (input wholesalers)
    \item There is much more; this is only *dedicated* six digit categories
    \item Vertical disintegration: farms outsourcing to specialists
    \item Sets up the research question
  \end{itemize}
}

\end{frame}

% % SLIDE 2: The Question and Why It Matters
\begin{frame}{The Question}

\textbf{Research question:}\\[0.3cm]
\begin{center}
\large Does the emergence of specialized upstream service industries\\
contribute to agricultural productivity growth?
\end{center}

\bigskip

\textbf{Why this matters:}
\begin{itemize}
    \item U.S. agricultural TFP growth is exceptional: \highlight{1.49\% annually} (1948--2021)
    \item Standard story: R\&D spillovers, technology adoption, input quality
    \item Missing piece: \redemph{organizational change} as a source of productivity
    \begin{itemize}
      \item Accounts for data imitations in hedonic adjustment
      \item Complements R\&D and technology adoption stories
    \end{itemize}
\end{itemize}

\bigskip

\textbf{Policy relevance:}
\begin{itemize}
    \item Understanding sources of productivity growth informs R\&D priorities
    \item Implications for rural labor markets and farm structure policy
\end{itemize}

\note{
  \begin{itemize}
    \item Big 5 \#1: What is the question?
    \item Big 5 \#2: Why is it important?
    \item Frame as filling a gap in the productivity literature
    \item Policy hook for applied audience
  \end{itemize}
}

\end{frame}

% % SLIDE 3: This Paper - Preview of contribution and findings
\begin{frame}{This Paper}

\textbf{What we do:}
\begin{itemize}
    \item First systematic estimates of relationship between crop services (1151) and yields
    \item Leverage restricted Census Bureau administrative data (LBD + Census of Ag)
    \item County-level panel: 2002--2017, four Census waves
\end{itemize}

\bigskip

\textbf{Preview of findings:}
\begin{enumerate}
    \item Upstream service employment \highlight{positively associated with yields}
    \item Effects are \highlight{heterogeneous}: strongest for corn and soybeans
    \item Effects stronger in counties with larger average farm size
    \item Suggestive evidence of \highlight{causal} relationship using longitudinal variation
\end{enumerate}

\bigskip

\textbf{Takeaway:} \textit{Organizational restructuring---not just technology---may be an important but overlooked source of agricultural productivity growth.}

\note{
  \begin{itemize}
    \item Big 5 \#3: What does existing literature say? (implicitly: not much on org change)
    \item Big 5 \#4: What is your contribution?
    \item First three findings are descriptive
    \item We also find suggestive evidence for causal effect
    \item Big 5 \#5: What is your key takeaway?
    \item Shapiro: Preview findings early---don't make them wait 45 minutes
  \end{itemize}
}

\end{frame}

% --------------------------------------------
\subsection{Background}
\label{subsec:background}

\begin{frame}{Not a New Question}

\textbf{From the 1974 Census of Agricultural Services:}

\bigskip

\begin{quote}
\small
``Until the 1940's, agriculture in America was largely self-reliant in regard to many production and harvesting practices now available from off-farm sources in the form of agricultural services. During the last three decades agricultural services have become an increasingly specialized industry. The technological and scientific changes in American agriculture have been directly related to the development of the agricultural service industry. A census of this industry is essential to provide facts necessary for:

\medskip

A. Broader view of today's farm production.\\
B. Better understanding and interpretation of long-term agricultural changes and trends.\\
C. More meaningful analysis of the interrelationships of agriculture and agricultural services.''
\end{quote}

\vfill
\hfill{\scriptsize --- U.S. Department of Commerce (1974), via Dunn \& Hueth (2017)}

\note{
  \begin{itemize}
    \item Direct quote from 1974 Census---they saw this coming
    \item Same motivation we have today, 50 years later
    \item Discontinued due to federal budget pressures in early 1980s
    \item Census Bureau lacked infrastructure/resources to restart
  \end{itemize}
}

\end{frame}

\begin{frame}{Not a New Question (cont.)}

\begin{columns}[c]
\begin{column}{0.52\textwidth}
\centering
\includegraphics[width=\textwidth]{figures/con_1978-2022.pdf}
\end{column}

\begin{column}{0.45\textwidth}
\small
\textbf{USDA recognized this 50+ years ago:}
\begin{itemize}
    \item Census of Ag Services: \highlight{1969, 1974, 1978}
    \item Discontinued when Census of Ag moved to NASS
\end{itemize}

\medskip

\textbf{A 45-year data gap:}
\begin{itemize}
    \item No systematic tracking of service providers
    \item Farm expenditure data, but not industry dynamics
\end{itemize}

\medskip

\textbf{Revived in 2022}
\begin{itemize}
    \item This paper: what can we learn from \textit{existing} administrative data?
\end{itemize}
\end{column}
\end{columns}

\note{
  \begin{itemize}
    \item Historical context---USDA saw this coming
    \item Data gap explains why this question hasn't been studied
    \item Sets up our contribution: use LBD + Census of Ag to fill the gap
    \item New Census of Ag Services will enable future research
  \end{itemize}
}

\end{frame}

\begin{frame}{Three Perspectives on Vertical Disintegration}

\begin{columns}[c]
\begin{column}{0.60\textwidth}
\centering
\includegraphics[width=\textwidth]{figures/input_shares.pdf}

\smallskip
{\scriptsize\textcolor{gray}{Source: Census of Agriculture, BEA Fixed Assets}}
\end{column}

\begin{column}{0.38\textwidth}
\small
\textbf{Three measures, same story:}

\smallskip

\begin{itemize}
    \item \textbf{Contract labor:} 11.6\% $\rightarrow$ \highlight{19.5\%}

    \smallskip

    \item \textbf{Custom work:} 16.4\% $\rightarrow$ \highlight{18.4\%}

    \smallskip

    \item \textbf{Machinery:} 6.1\% $\rightarrow$ \highlight{14.1\%}
\end{itemize}

\medskip

{\footnotesize\textit{Work and capital moving off the farm}}
\end{column}
\end{columns}

\note{
  \begin{itemize}
    \item Figure 1 from the paper
    \item Three independent data sources, same story
    \item Contract labor from Census of Ag
    \item Machinery ownership from BEA Fixed Asset Tables
    \item Massive growth in services, flat farm sales
  \end{itemize}
}

\end{frame}

\begin{frame}{Why Would Disintegration Boost Productivity?}

\textbf{Classic economic logic:}
\begin{itemize}
    \item \textbf{Specialization} --- Farms focus on core competencies; specialists develop expertise
    \item \textbf{Scale economies} --- One custom harvester can serve many farms
    \item \textbf{Technology adoption} --- Specialists can justify expensive, cutting-edge equipment
\end{itemize}

\bigskip

\textbf{Additional mechanisms:}
\begin{itemize}
    \item \textbf{Risk smoothing} --- Service firms diversify across geography and seasons
    \item \textbf{Labor market efficiency} --- Skilled operators matched to equipment
    \item \textbf{Knowledge spillovers} --- Specialists transfer best practices across farms
\end{itemize}

\bigskip

\textit{If disintegration enables these efficiencies, we should see a positive relationship between upstream service activity and farm productivity.}

\note{
  \begin{itemize}
    \item Intuitive appetizer---give them a mental model before regressions
    \item Smith/Ricardo specialization logic
    \item Sets up the empirical test
    \item Counterargument: transaction costs, coordination problems (address in discussion)
  \end{itemize}
}

\end{frame}

% --------------------------------------------
\subsection{Data and Methodology}
\label{subsec:data}

\begin{frame}{Data Sources}

\textbf{Farm productivity (Census of Agriculture, 2002--2022):}
\begin{itemize}
    \item County-level crop sales, harvested acreage, yields
    \item Corn, soybeans, wheat (bushels/acre); all crops (dollars/acre)
    \item Farm structure: number of farms, land distribution, HHI
\end{itemize}

\bigskip

\textbf{Service provider activity (Longitudinal Business Database, 2002-2017):}
\begin{itemize}
    \item Restricted-access establishment-level microdata
    \item NAICS 1151: Support Activities for Crop Production
    \item \highlight{County-level payroll} as proxy for service intensity
    \item Derived from IRS Forms 941/943 (payroll tax filings)
\end{itemize}

\bigskip

\textbf{Why LBD?} Public data (QCEW) suppressed for \highlight{75\%} of counties with crop services---yet these counties produce \highlight{half of U.S. crops}

\note{
  \begin{itemize}
    \item Census of Ag: public county tables from NASS
    \item LBD: restricted access through FSRDC network
    \item Payroll better than employment (March 12 employment misses seasonality)
    \item Key contribution: overcoming data suppression problem
  \end{itemize}
}

\end{frame}

\begin{frame}{The Data Suppression Problem}

\begin{columns}[T]
\begin{column}{0.48\textwidth}
\textbf{QCEW Data Availability (2017):}

\smallskip

\begin{tabular}{lr}
\toprule
Category & Counties \\
\midrule
No services & 1,185 \\
Unsuppressed & 481 \\
\redemph{Suppressed} & \redemph{1,408} \\
\midrule
Total & 3,074 \\
\bottomrule
\end{tabular}

\bigskip

74.5\% of counties with crop services have suppressed data
\end{column}

\begin{column}{0.48\textwidth}
\textbf{But suppressed counties matter:}

\smallskip

\begin{tabular}{lrr}
\toprule
 & Unsupp. & Supp. \\
\midrule
Estab. & 64\% & 36\% \\
Payroll & 83\% & 17\% \\
\highlight{Crop sales} & \highlight{50\%} & \highlight{50\%} \\
\bottomrule
\end{tabular}

\bigskip

Half of U.S. crop production occurs in counties with suppressed service data
\end{column}
\end{columns}

\bigskip

\textbf{Solution:} Use restricted LBD microdata to construct unsuppressed county-level payroll measures

\note{
  \begin{itemize}
    \item Table 1 from the paper
    \item Suppression for disclosure avoidance (few establishments)
    \item Can't study services-productivity relationship with public data
    \item LBD access through FSRDC is key contribution
  \end{itemize}
}

\end{frame}

\begin{frame}{Empirical Strategy: Cross-Sectional}

\textbf{Basic specification (repeated cross-sections, 2002--2017):}

\begin{equation*}
\ln(\text{Yield}) = \beta_0 + \beta_1 \ln(\text{Payroll}) + \beta_2 \ln(\text{Land}) + \beta_3 \ln(\text{AvgLand}) + \beta_4 \text{HHI} + \varepsilon
\end{equation*}

\bigskip

\begin{itemize}
    \item All variables constructed at county-year, Census years (2002, 2007, 2012, 2017)
    \item \textbf{Outcome:} Yield (bu/acre for corn, soy, wheat; \$/acre for all crops)
    \item \textbf{Key regressor:} Log payroll in NAICS 1151
    \item \textbf{Controls:} Total harvested acres, mean farm size, land concentration (HHI)
\end{itemize}

\bigskip

\textbf{Interpretation:} $\beta_1$ = elasticity of yield with respect to service payroll

\bigskip

\textit{Estimate separately by year and crop to allow relationship to vary}

\note{
  \begin{itemize}
    \item Log-log specification for elasticity interpretation
    \item Controls address scale and structure confounds
    \item Repeated cross-section (not panel FE) to see how relationship evolves
    \item Hypothesis: $\beta_1 > 0$ if services boost productivity
  \end{itemize}
}

\end{frame}

\begin{frame}{Empirical Strategy: Longitudinal}

\textbf{Addressing simultaneity:} Does service activity cause productivity, or do productive counties attract services?

\bigskip

\textbf{Dynamic specification:}

\begin{equation*}
\Delta \ln(\text{Yield}_{c,t+5}) = \beta_0 + \beta_1 \ln(\text{Yield}_{ct}) + \beta_2 \ln(\text{Payroll}_{ct}) + \mathbf{X}_{ct}'\boldsymbol{\gamma} + \varepsilon_{ct}
\end{equation*}

\bigskip

\begin{itemize}
    \item \textbf{Outcome:} 5-year yield \textit{growth} (log difference)
    \item \textbf{Key regressor:} \textit{Base-year} service payroll
    \item \textbf{Control for base-year yield:} Addresses reverse causality, mean reversion
\end{itemize}

\bigskip

\textbf{Interpretation:} $\beta_2 > 0$ means counties with larger service sectors experience \textit{faster subsequent yield growth}, conditional on initial productivity

\note{
  \begin{itemize}
    \item Key innovation: use lagged payroll to predict future growth
    \item Controls for initial productivity level
    \item Not full causal identification, but addresses main endogeneity concern
    \item Equation 3 from the paper
  \end{itemize}
}

\end{frame}

% --------------------------------------------
\subsection{Results}
\label{subsec:results}

\begin{frame}{Cross-Sectional Results: Services and Yields}

\textbf{Key finding:} Persistent positive relationship between service payroll and yields

\bigskip

\begin{center}
\begin{tabular}{lcccc}
\toprule
Year & All Crops & Corn & Soybeans & Wheat \\
\midrule
2002 & 0.228*** & 0.035*** & 0.008* & 0.043*** \\
2007 & 0.196*** & 0.027*** & 0.009 & 0.048*** \\
2012 & 0.194*** & 0.063*** & 0.028*** & 0.033*** \\
2017 & 0.193*** & 0.020*** & 0.007 & 0.024*** \\
\bottomrule
\end{tabular}
\end{center}

{\small\textit{Elasticity of yield w.r.t. crop services payroll. Controls: land, avg. farm size, HHI.}}

\bigskip

\textbf{Interpretation:} 1\% increase in service payroll $\rightarrow$ \highlight{0.02--0.06\%} higher corn yields

\note{
  \begin{itemize}
    \item Table 3 from the paper
    \item All crops uses \$/acre; individual crops use bu/acre
    \item Soybeans weakest association---less labor intensive?
    \item Coefficients smaller for individual crops (payroll affects all crops)
  \end{itemize}
}

\end{frame}

\begin{frame}[label=farm_scale]{The Role of Farm Scale}

\textbf{Does the services-yield relationship vary with farm structure?}

\bigskip

\begin{columns}[T]
\begin{column}{0.48\textwidth}
\textbf{By total harvested land:}
\begin{itemize}
    \item \textbf{Wheat:} Effect \highlight{stronger} in larger counties
    \item \textbf{Corn:} Effect \highlight{weaker} as acreage increases
    \item \textbf{Soybeans:} No significant pattern
\end{itemize}

\medskip

\textit{Suggests different mechanisms by crop}
\end{column}

\begin{column}{0.48\textwidth}
\textbf{By average farm size:}
\begin{itemize}
    \item Effect often \highlight{positive}
    \item Service providers have more impact in counties with larger farms
\end{itemize}

\medskip

\textit{Consistent with disintegration story: large farms outsource more}
\end{column}
\end{columns}

\bigskip

\textbf{Implication:} One-size-fits-all analysis obscures important heterogeneity

\bigskip

\hfill
\hyperlink{fig3_panel_a}{\beamergotobutton{Panel A: Total Land}}
\hspace{0.5cm}
\hyperlink{fig3_panel_b}{\beamergotobutton{Panel B: Avg Farm Size}}

\note{
  \begin{itemize}
    \item In cross-section, interact payroll with Land and AvgLand
    \item Figure 3 from the paper shows marginal effects
    \item Wheat: scale economies in services matter more
    \item Corn: diminishing returns at larger scale
    \item Larger effects in counties with larger farms
    \item Will come back to interpretation of magnitude in later slide
  \end{itemize}
}

\end{frame}

\begin{frame}{Longitudinal Results: Services and Yield \textit{Growth}}

\textbf{Question:} Do larger service sectors predict \textit{faster} yield growth?

\bigskip

\begin{center}
\begin{tabular}{lcccc}
\toprule
Period & All Crops & Corn & Soybeans & Wheat \\
\midrule
2002--2007 & 0.006 & 0.009* & 0.006 & 0.020*** \\
2007--2012 & 0.030*** & 0.049*** & 0.028*** & 0.007* \\
2012--2017 & 0.035*** & $-$0.004 & $-$0.003 & 0.000 \\
2017--2022 & $-$0.001 & 0.017*** & 0.005 & 0.001 \\
\bottomrule
\end{tabular}
\end{center}

{\small\textit{Coefficient on base-year payroll, controlling for base-year yield and farm structure.}}

\bigskip

\textbf{Key pattern:} Positive effects in early periods, \redemph{weakening over time}

\note{
  \begin{itemize}
    \item Table 4 from the paper
    \item Addresses simultaneity: does service activity \textit{cause} productivity?
    \item Controlling for base-year yield addresses reverse causality
    \item Weakening suggests diminishing returns as sector matures
  \end{itemize}
}

\end{frame}

\begin{frame}{Economic Magnitude: How Much Does It Matter?}

\textbf{Thought experiment:} Compare counties at 50th vs. 75th percentile of services

\bigskip

\begin{center}
\begin{tabular}{lcccc}
\toprule
Period & All Crops & Corn & Soybeans & Wheat \\
\midrule
2002--2007 & --- & 1.2 & --- & 2.7 \\
2007--2012 & 4.0 & 6.6 & 3.5 & 1.0 \\
2012--2017 & 4.6 & --- & --- & --- \\
2017--2022 & --- & 2.0 & --- & --- \\
\bottomrule
\end{tabular}
\end{center}

{\small\textit{Difference in 5-year productivity growth rate (percentage points)}}

\bigskip

\textbf{Example:} A county at 75th percentile of crop services in 2007 experienced \highlight{6.6 percentage points} higher corn yield growth over 2007--2012 than a county at 50th percentile

\bigskip

\textit{Modest but economically meaningful differences}

\note{
  \begin{itemize}
    \item Table 5 from the paper
    \item Makes elasticities concrete
    \item 6.6 pp difference over 5 years is substantial
    \item Blank cells = coefficient not statistically significant
  \end{itemize}
}

\end{frame}

\begin{frame}{Results Summary}

\begin{enumerate}
    \item \textbf{Persistent positive relationship}\\
    Counties with larger crop services sectors have higher yields (2002--2017)

    \bigskip

    \item \textbf{Heterogeneity matters}\\
    Effects vary by crop type and farm scale; effects larger in counties with larger avg. farm size

    \bigskip

    \item \textbf{Dynamic evidence supports causality}\\
    Larger services sectors predict \textit{faster subsequent} yield growth

    \bigskip

    \item \textbf{But relationship is weakening}\\
    Consistent with maturing industry, exhausted scale economies
\end{enumerate}

\bigskip

\textbf{Bottom line:} Organizational restructuring---vertical disintegration---appears to be an \highlight{overlooked source} of yield growth

\note{
  \begin{itemize}
    \item Wrap up results before moving to discussion
    \item Key message: disintegration contributes to productivity
    \item But effects may be diminishing as sector matures
    \item Sets up discussion of implications
  \end{itemize}
}

\end{frame}

% --------------------------------------------
\subsection{Discussion}
\label{subsec:discussion}

\begin{frame}{Why Is the Relationship Weakening?}

\textbf{Observed pattern:} Services-yield relationship strongest in 2007--2012, fading since

\bigskip

\textbf{Possible explanations:}

\begin{enumerate}
    \item \textbf{Maturing industry}
    \begin{itemize}
        \item Entry and exit rates declining since 1990s
        \item ``Creative destruction'' drives innovation; less dynamism $\rightarrow$ slower gains
    \end{itemize}

    \bigskip

    \item \textbf{Exhausted scale economies}
    \begin{itemize}
        \item Early adopters captured largest gains
    \end{itemize}

    \bigskip

    \item \textbf{Farm consolidation}
    \begin{itemize}
        \item Largest farms increasingly self-sufficient
        \item Service providers working with fewer, larger clients
    \end{itemize}
\end{enumerate}

\note{
  \begin{itemize}
    \item Important to address why effects are weakening
    \item Industry dynamism declining---business dynamics literature
    \item Consistent with Stigler: scale economies eventually exhausted
    \item Farm consolidation may be shifting the equilibrium
  \end{itemize}
}

\end{frame}

\begin{frame}{Industry Dynamism Is Declining}

\textbf{Evidence from the LBD:}

\bigskip

\begin{columns}[T]
\begin{column}{0.48\textwidth}
\textbf{Entry and exit rates falling:}
\begin{itemize}
    \item Business entry/exit rates in crop services declining since 1990s
    \item Less ``creative destruction''
    \item Fewer new entrants bringing innovations
\end{itemize}

\bigskip

\textit{When the industry was young, new firms drove productivity gains}
\end{column}

\begin{column}{0.48\textwidth}
\textbf{Establishment counts stable:}
\begin{itemize}
    \item Mean establishments/county: 5.75 (2002) $\rightarrow$ 5.19 (2017)
    \item Slight decline, not growth
    \item Industry has \highlight{matured}
\end{itemize}

\bigskip

\textit{Sector reached equilibrium size}
\end{column}
\end{columns}

\bigskip

\textbf{Implication:} Biggest productivity gains may have come from early industry growth; mature industry offers fewer marginal improvements

\note{
  \begin{itemize}
    \item Business dynamics literature: declining dynamism across economy
    \item Crop services sector fits this pattern
    \item Creative destruction theory: entry/exit drives innovation
    \item May explain why relationship weakened post-2012
  \end{itemize}
}

\end{frame}

\begin{frame}{Local Market Concentration: A Surprising Finding}

\textbf{Contrast with farming sector:}

\bigskip

\begin{columns}[T]
\begin{column}{0.48\textwidth}
\textbf{Farms are consolidating:}
\begin{itemize}
    \item Number of farms declining
    \item Average farm size increasing
    \item Land concentration rising
\end{itemize}

\bigskip

\textit{Well-documented trend}
\end{column}

\begin{column}{0.48\textwidth}
\textbf{Services are NOT consolidating:}
\begin{itemize}
    \item Largest provider's payroll share:
    \item 2002: 73.0\%
    \item 2017: 73.9\%
    \item \highlight{Essentially unchanged}
\end{itemize}

\bigskip

\textit{Unexpected stability}
\end{column}
\end{columns}

\bigskip

\textbf{Why does this matter?}
\begin{itemize}
    \item Service providers working with fewer, larger farms
    \item But local market power hasn't increased
    \item Suggests competitive pressure remains
\end{itemize}

\note{
  \begin{itemize}
    \item Table 2 from the paper
    \item Surprising finding: no concentration increase
    \item Contrasts with farm consolidation, processor consolidation
    \item May reflect geographic nature of service provision
    \item Important for market power concerns in Discussion
  \end{itemize}
}

\end{frame}

\begin{frame}{Implications for Productivity Measurement}

\textbf{How does this affect TFP accounting?}

\bigskip

% \begin{columns}[T]
% \begin{column}{0.48\textwidth}
\textbf{Quality mismeasurement:}
\begin{itemize}
    \item Services = substitute for own-capital
    \item Same price index applied to both
    \item But: machinery managed by specialists is \highlight{more productive}
    \item $\Rightarrow$ Service input quality \textit{understated}
    \item $\Rightarrow$ TFP growth \textit{overstated}
\end{itemize}
%\end{column}

% \begin{column}{0.48\textwidth}
% \textbf{Market power concerns:}
% \begin{itemize}
%     \item High local concentration in services
%     \item Contract workers paid less than hired workers
%     \item If monopsony: wages $<$ marginal product
%     \item $\Rightarrow$ Labor share understated
%     \item $\Rightarrow$ Another source of TFP bias
% \end{itemize}
% \end{column}
% \end{columns}

\bigskip

\textbf{Takeaway:} Vertical disintegration creates measurement challenges not fully addressed in current TFP frameworks

\note{
  \begin{itemize}
    \item Links to productivity measurement literature
    \item Quality adjustment is key issue
    \item Wang \& Loduca (2025): contract workers paid less
    \item Suggests some ``TFP growth'' may be mismeasured input growth
  \end{itemize}
}

\end{frame}

\begin{frame}{Connection to Manufacturing TFP Research}

\textbf{Recent work reveals linked measurement challenges:}

\bigskip

\textbf{Atalay et al. (2025):} ``Why Is Manufacturing Productivity Growth So Low?''
\begin{itemize}
    \item Producer price indices \highlight{understate quality improvements} in durable goods
    \item Leads to systematic \textit{understatement} of manufacturing TFP growth
\end{itemize}

\bigskip

\textbf{Connection to our findings:}
\begin{itemize}
    \item Manufacturers supply precision equipment, GPS-guided machinery to service providers
    \item Understated quality in manufacturing output prices $\Leftrightarrow$ understated quality in farm input prices
    \item \highlight{Same unmeasured quality differential} flows through intermediate goods markets
\end{itemize}

\bigskip

\textbf{Implication:} Comprehensive productivity accounting requires coordinated attention to quality adjustment across \textit{linked industries}

\note{
  \begin{itemize}
    \item Atalay et al. NBER WP 34264
    \item Nice connection: same problem from opposite ends of supply chain
    \item Manufacturing understates output quality $\rightarrow$ Ag understates input quality
    \item Suggests need for integrated approach to productivity measurement
  \end{itemize}
}

\end{frame}

\begin{frame}{Broader Implications: Agricultural Development}

\textbf{Why this matters beyond U.S. agriculture:}

\bigskip

\begin{columns}[T]
\begin{column}{0.48\textwidth}
\textbf{Downstream value chains:}
\begin{itemize}
    \item Large literature on connecting farmers to processors and consumers
    \item Market access, contract farming, supply chain coordination
    \item (Reardon 2015; Barrett et al. 2022)
\end{itemize}
\end{column}

\begin{column}{0.48\textwidth}
\textbf{Technology adoption:}
\begin{itemize}
    \item Extensive research on farm-level adoption decisions
    \item Information, risk, heterogeneity
    \item (Feder et al. 1985; Suri 2011)
\end{itemize}
\end{column}
\end{columns}

\bigskip

\textbf{Missing piece: Upstream market development}
\begin{itemize}
    \item Technology adoption depends on \highlight{existence of input markets}
    \item Specialized service providers enable technology access at scale
    \item U.S. experience: services sector grew \textit{alongside} productivity gains
\end{itemize}

\bigskip

\textit{Upstream value chain ``thickening'' deserves attention in development contexts}

\note{
  \begin{itemize}
    \item Relevant for ISU's international focus
    \item Downstream value chains well-studied
    \item Upstream market development less so
    \item U.S. case suggests services enable technology adoption
    \item Policy implication: support input market development
  \end{itemize}
}

\end{frame}

\begin{frame}{Limitations and Future Directions}

\textbf{Limitations:}
\begin{itemize}
    \item County-level analysis---can't observe farm-level outsourcing decisions
    \item Payroll is proxy for service intensity, not direct output measure
    \item Cannot fully separate services effect from technology adoption
    \item Equilibrium relationships, not structural causal estimates
\end{itemize}

\bigskip

\textbf{Future research directions:}
\begin{itemize}
    \item \textbf{2022 Census of Ag Services:} First since 1978---new microdata opportunities
    \item \textbf{Farm-level analysis:} Link service use to individual farm outcomes
    \item \textbf{Broader supply chain:} Extend to fertilizer, chemicals, equipment sectors
    \item \textbf{Quality adjustment:} Develop crop-specific service price indices
\end{itemize}

\note{
  \begin{itemize}
    \item Honest about limitations
    \item 2022 Census is a big deal---revived after 44 years
    \item Sets up future research agenda
    \item Connects to position at ISU---supply chain focus
  \end{itemize}
}

\end{frame}

% --------------------------------------------
\subsection{Conclusion}
\label{subsec:conclusion}

\begin{frame}{Conclusion}

\textbf{What we asked:}\\
Does the emergence of specialized upstream service industries contribute to agricultural productivity growth?

\bigskip

\textbf{What we found:}
\begin{itemize}
    \item \highlight{Yes}---counties with larger crop services sectors have higher yields
    \item Effect is heterogeneous: varies by crop, scale, and time
    \item Longitudinal evidence consistent with causal interpretation
    \item But gains appear to be diminishing as the sector matures
\end{itemize}

\bigskip

\textbf{Why it matters:}
\begin{itemize}
    \item Organizational change is an \highlight{overlooked source} of productivity growth
    \item Standard TFP frameworks may not fully capture these dynamics
    \item Understanding the farm-services boundary is essential for policy
\end{itemize}

\note{
  \begin{itemize}
    \item Circle back to opening question
    \item Clear answer: yes, with nuance
    \item Broader implication for how we think about productivity
  \end{itemize}
}

\end{frame}

\begin{frame}{Key Takeaway}

\begin{center}
\Large
\textit{The ``boundary of the farm'' is not fixed.}

\bigskip

\textit{As specialized service industries emerge,\\
work moves off the farm---and productivity rises.}

\bigskip

\textit{This organizational restructuring deserves\\
greater attention in productivity research.}
\end{center}

\note{
  \begin{itemize}
    \item Big 5 \#5: What is the key takeaway?
    \item Memorable framing for the audience
    \item Connects to title of the paper
  \end{itemize}
}

\end{frame}

\begin{frame}[plain]

\vfill

\begin{center}
{\Large\textbf{Thank You}}

\bigskip
\bigskip

Brent Hueth\\
\textit{USDA Economic Research Service}\\
\smallskip
brent.hueth@usda.gov

\bigskip
\bigskip

Joint work with:\\
Anton Babkin (UW-Madison)\\
Richard A. Dunn (Sandhill Consulting)

\bigskip
\bigskip

{\small Slides available upon request}
\end{center}

\vfill

\end{frame}

% ============================================
% APPENDIX: BACKUP SLIDES
% ============================================

\appendix

% Panel A: Marginal Effects by Total Harvested Land
\begin{frame}[label=fig3_panel_a]{Marginal Effects: By Total Harvested Farmland}

%\textbf{Figure 3, Panel A} 
\hfill \hyperlink{farm_scale}{\fcolorbox{darkblue}{white}{\textcolor{darkblue}{\footnotesize $\triangleleft$ Back to Results}}}

\smallskip

\begin{center}
\includegraphics[width=0.70\textwidth]{figures/fig3_panel_a.pdf}
\end{center}

\note{
  \begin{itemize}
    \item Figure 3 Panel A from the paper
    \item Shows marginal effects at different levels of total county harvested farmland
    \item Wheat: larger counties see stronger services-yield relationship
    \item Corn: relationship weakens with scale---diminishing returns?
  \end{itemize}
}

\end{frame}

% Panel B: Marginal Effects by Average Farm Size
\begin{frame}[label=fig3_panel_b]{Marginal Effects: By Average Farm Size}

%\textbf{Figure 3, Panel B} 
\hfill \hyperlink{farm_scale}{\fcolorbox{darkblue}{white}{\textcolor{darkblue}{\footnotesize $\triangleleft$ Back to Results}}}

\smallskip

\begin{center}
\includegraphics[width=0.70\textwidth]{figures/fig3_panel_b.pdf}
\end{center}

\note{
  \begin{itemize}
    \item Figure 3 Panel B from the paper
    \item Shows marginal effects at different levels of average farm size
    \item Positive relationship: counties with larger farms benefit more from services
    \item Supports disintegration story: large farms are the ones outsourcing
  \end{itemize}
}

\end{frame}

% ============================================
% END
% ============================================

\end{document}
